\documentclass[10pt, landscape]{article}
\usepackage[scaled=0.92]{helvet}
\usepackage{calc}
\usepackage{multicol}
\usepackage[a4paper,margin=3mm,landscape]{geometry}
\usepackage{amsmath,amsthm,amsfonts,amssymb}
\usepackage{color,graphicx,overpic}
\usepackage{hyperref}
\usepackage{newtxtext} 
\usepackage{enumitem}
\usepackage[table]{xcolor}
\usepackage{mathtools}
\setlist{nosep}
\usepackage{subfig}
\usepackage{listings}

% Define Rust language for listings package
\lstdefinelanguage{Rust}{
  morekeywords={let, mut},
  morecomment=[l]{//},
  morecomment=[s]{/*}{*/},
  morestring=[b]",
}

% for including images
\graphicspath{ {./images/} }


\pdfinfo{
  /Title (CS3231.pdf)
  /Creator (TeX)
  /Producer (pdfTeX 1.40.0)
  /Author (Pei Cheng Yi)
  /Subject (CS3231)
  /Keywords (CS3231, nus,cheatsheet,pdf)}

% Turn off header and footer
\pagestyle{empty}

\newenvironment{tightcenter}{%
  \setlength\topsep{0pt}
  \setlength\parskip{0pt}
  \begin{center}
}{%
  \end{center}
}

% redefine section commands to use less space
\makeatletter
\renewcommand{\section}{\@startsection{section}{1}{0mm}%
                                {-1ex plus -.5ex minus -.2ex}%
                                {0.5ex plus .2ex}%x
                                {\normalfont\large\bfseries}}
\renewcommand{\subsection}{\@startsection{subsection}{2}{0mm}%
                                {-1explus -.5ex minus -.2ex}%
                                {0.5ex plus .2ex}%
                                {\normalfont\normalsize\bfseries}}
\renewcommand{\subsubsection}{\@startsection{subsubsection}{3}{0mm}%
                                {-1ex plus -.5ex minus -.2ex}%
                                {1ex plus .2ex}%
                                {\normalfont\small\bfseries}}%
\renewcommand{\familydefault}{\sfdefault}
\renewcommand\rmdefault{\sfdefault}
%  makes nested numbering (e.g. 1.1.1, 1.1.2, etc)
\renewcommand{\labelenumii}{\theenumii}
\renewcommand{\theenumii}{\theenumi.\arabic{enumii}.}
\renewcommand\labelitemii{•}
\renewcommand\labelitemiii{•}
%  convenient absolute value symbol
\newcommand{\abs}[1]{\vert #1 \vert}
%  convenient floor and ceiling
\newcommand{\floor}[1]{\lfloor #1 \rfloor}
\newcommand{\ceil}[1]{\lceil #1 \rceil}
%  convenient modulo
\newcommand{\Mod}[1]{\ \mathrm{mod}\ #1}
%  for logical not operator, iff symbol, convenient "if/then"
\renewcommand{\lnot}{\mathord{\sim}}
\let\then\Rightarrow
\let\Then\Rightarrow
%  vectors
\newcommand{\vv}[1]{\boldsymbol{#1}}
\newcommand{\VV}[1]{\overrightarrow{#1}}
%  column vector
\newcommand{\cvv}[1]{\left(\begin{smallmatrix}#1\end{smallmatrix}\right)}
\newcommand{\code}[1]{\textcolor{myblue}{\texttt{#1}}}
\newcommand\bggreen{\cellcolor{green!10}}

\makeatother
\definecolor{myblue}{cmyk}{1,.72,0,.38}
\everymath\expandafter{\the\everymath \color{myblue}}
% Define BibTeX command
\def\BibTeX{{\rm B\kern-.05em{\sc i\kern-.025em b}\kern-.08em
    T\kern-.1667em\lower.7ex\hbox{E}\kern-.125emX}}

% Don't print section numbers
\setcounter{secnumdepth}{0}

\setlength{\parindent}{0pt}
\setlength{\parskip}{0pt plus 0.5ex}
%% this changes all items (enumerate and itemize)
\setlength{\leftmargini}{0.5cm}
\setlength{\leftmarginii}{0.4cm}
\setlength{\leftmarginiii}{0.5cm}
\setlist[enumerate,1]{leftmargin=2mm,labelindent=1mm,labelsep=1mm}
\setlist[itemize,1]{leftmargin=2mm,labelindent=1mm,labelsep=1mm}
\setlist[itemize,2]{leftmargin=3mm,labelindent=1mm,labelsep=1mm}
\setlist[itemize,3]{leftmargin=3mm,labelindent=1mm,labelsep=1mm}

%My Environments
\newtheorem{example}[section]{Example}
% -----------------------------------------------------------------------

\begin{document}
\raggedright
\footnotesize
\begin{multicols}{4}


% multicol parameters
% These lengths are set only within the two main columns
\setlength{\columnseprule}{0.25pt}
\setlength{\premulticols}{1pt}
\setlength{\postmulticols}{1pt}
\setlength{\multicolsep}{1pt}
\setlength{\columnsep}{2pt}

\begin{center}
    \fbox{%
        \parbox{0.8\linewidth}{\centering \textcolor{black}{
            {\Large\textbf{CS3231}}
            \\ \normalsize{AY23/24 S1}}
            \\ {\footnotesize \textcolor{myblue}{github.com/SeekSaveServe}}
        }%
    }
\end{center}

% LECTURES

\section{Lectures}

\subsection{L1: Preliminaries, DFA}
\textbf{Types of Proof}
\begin{itemize}
  \item Deductive Proofs
  \item Modus Ponens, $A \rightarrow B$
  \item Proof by contradiction
  \item Counter-example
  \item Equivalence (iff)
  \item Converse, $A \rightarrow B$ and $B \rightarrow A$ shows iff and equivalence 
  \item Inductive Proofs
  \item \begin{proof}
    Prove base case \\
    Assume true for $n=k$\\
    Prove true for $n=k+1$ 
  \end{proof}
  \item Structural Induction: if a claim holds true for a tree of height $k$, then it holds true for a tree of height $k+1$
  \item Mutual Induction: Showing several claims to be true simultaneously
  \item Diagonalisation: Showing that a set is uncountable by demonstrating a contradiction to the assumption that it is possible to enumerate all elements of the set (countable). This can be done by changing the $i^{th}$ digit of the $i^{th}$ element of the enumeration, ensuring that the new element differs from existing elements in the enumeration by at least 1 digit.
\end{itemize}

\textbf{Central Concepts of Automata Theory}
\begin{itemize}
  \item \textbf{Alphabet $\Sigma$}: finite non-empty set of symbols (e.g. $\{0,1\}$)
  \item \textbf{String}: finite sequence of symbols from $\Sigma$ (e.g. $0101$)
  \item \textbf{Empty String $\epsilon$}: string with no symbols
  \item \textbf{Length of String $|w|$}: number of symbols in string $w$
  \item \textbf{Powers of n Alphabet $\Sigma^n$}: set of all strings of length $n$ over $\Sigma$ (e.g. $\{0,1\}^2 = \{00,01,10,11\}$)
  \item \textbf{Concatenation of Strings} $w_1 = 0101, w_2 = 1010, w_1w_2 = w_1 . w_2 = 01011010$
  \item \textbf{Substring} $ab$ is a substring of $babaa$ but $bb$ is not 
  \item \textbf{Subsequence} $bba$ is a subsequence of $babaa$ but $abb$ is not. Relative order matters, can skip.
  \item \textbf{Language $L$} A set of strings over $\Sigma$ 
\end{itemize}

\textbf{Strict Definitiion of a Language}
\begin{itemize}
  \item A language is strictly defined. When a turing machine accepts a language, it accepts exactly that language, not a superset or subset
  \item $L=\{x : x$ is a binary representation of a prime number$\}$ is not clearly defined since $011$ and $11$ are both binary representations of $3$
  \item $L_1 . L_2 = \{ xy : x \in L_1,, y \in L_2\}$
  \item $L^* = \{ x_1x_2...x_n : x_1, x_2..., x_n \in L, n \in \mathbb{N} \}$ since n can be 0, $\epsilon \in L^*$
  \item $L^* = \{ \epsilon \} \cup L \cup L^2 \cup L^3 \cup ...$
  \item $L^+ \{ x_1x_2...x_n : x_1, x_2..., x_n \in L, n \ge 1 \}$
  \item $L^+ = L \cup L^2 \cup L^3 \cup ...$, note that it may or may not include $\epsilon$ since $\epsilon$ can be included in $L$
  \item Number of srings over any fixed fininte alphabet is countable 
  \item Number of languages over any non-empty alphabet is uncountable
\end{itemize}

\textbf{Finite Automata}
\begin{itemize}
  \item Regular language is accepted by a finite automata 
  \item E.g. On-Off Switch. Inputs toggles the state
\end{itemize}

\textbf{Deterministic Finite Automata (DFA)}
\begin{itemize}
  \item A DFA is a 5-tuple $A = (Q, \Sigma, \delta, q_0, F)$ where
  \item $Q$ is a finite set of states
  \item $\Sigma$ is a finite alphabet
  \item $\delta(q_i, x | q_i \in Q, x \in \Sigma): Q \times \Sigma \rightarrow Q$ is the transition function
  \item $q_0 \in Q$ is the start state
  \item $F \subseteq Q$ is the set of accept states
\end{itemize}
% Tutorials


% Misc

\end{multicols}
\end{document}