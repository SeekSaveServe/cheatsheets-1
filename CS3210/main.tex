\documentclass[10pt, landscape]{article}
\usepackage[scaled=0.92]{helvet}
\usepackage{calc}
\usepackage{graphicx}
\usepackage{multicol}
\usepackage{ifthen}
\usepackage[a4paper,margin=3mm,landscape]{geometry}
\usepackage{amsmath,amsthm,amsfonts,amssymb}
\usepackage{color,graphicx,overpic}
\usepackage{hyperref}
\usepackage{newtxtext} 
\usepackage{enumitem}
\usepackage{graphicx}
\usepackage[table]{xcolor}
\usepackage{mathtools}
\usepackage[document]{ragged2e}
\usepackage{listings}
\setlist{nosep}
\usepackage{subfig}
\usepackage{listings}

% Define Rust language for listings package
\lstdefinelanguage{Rust}{
  morekeywords={let, mut},
  morecomment=[l]{//},
  morecomment=[s]{/*}{*/},
  morestring=[b]",
}

% for including images
\graphicspath{ {./images/} }


\pdfinfo{
  /Title (CS3211.pdf)
  /Creator (TeX)
  /Producer (pdfTeX 1.40.0)
  /Author (Pei Cheng Yi)
  /Subject (CS3211)
  /Keywords (CS3211, nus,cheatsheet,pdf)}

% Turn off header and footer
\pagestyle{empty}

\newenvironment{tightcenter}{%
  \setlength\topsep{0pt}
  \setlength\parskip{0pt}
  \begin{center}
}{%
  \end{center}
}

% redefine section commands to use less space
\makeatletter
\renewcommand{\section}{\@startsection{section}{1}{0mm}%
                                {-1ex plus -.5ex minus -.2ex}%
                                {0.5ex plus .2ex}%x
                                {\normalfont\large\bfseries}}
\renewcommand{\subsection}{\@startsection{subsection}{2}{0mm}%
                                {-1explus -.5ex minus -.2ex}%
                                {0.5ex plus .2ex}%
                                {\normalfont\normalsize\bfseries}}
\renewcommand{\subsubsection}{\@startsection{subsubsection}{3}{0mm}%
                                {-1ex plus -.5ex minus -.2ex}%
                                {1ex plus .2ex}%
                                {\normalfont\small\bfseries}}%
\renewcommand{\familydefault}{\sfdefault}
\renewcommand\rmdefault{\sfdefault}
%  makes nested numbering (e.g. 1.1.1, 1.1.2, etc)
\renewcommand{\labelenumii}{\theenumii}
\renewcommand{\theenumii}{\theenumi.\arabic{enumii}.}
\renewcommand\labelitemii{•}
\renewcommand\labelitemiii{•}
%  convenient absolute value symbol
\newcommand{\abs}[1]{\vert #1 \vert}
%  convenient floor and ceiling
\newcommand{\floor}[1]{\lfloor #1 \rfloor}
\newcommand{\ceil}[1]{\lceil #1 \rceil}
%  convenient modulo
\newcommand{\Mod}[1]{\ \mathrm{mod}\ #1}
%  for logical not operator, iff symbol, convenient "if/then"
\renewcommand{\lnot}{\mathord{\sim}}
\let\then\Rightarrow
\let\Then\Rightarrow
%  vectors
\newcommand{\vv}[1]{\boldsymbol{#1}}
\newcommand{\VV}[1]{\overrightarrow{#1}}
%  column vector
\newcommand{\cvv}[1]{\left(\begin{smallmatrix}#1\end{smallmatrix}\right)}
\newcommand{\code}[1]{\textcolor{myblue}{\texttt{#1}}}
\newcommand\bggreen{\cellcolor{green!10}}

\makeatother
\definecolor{myblue}{cmyk}{1,.72,0,.38}
\everymath\expandafter{\the\everymath \color{myblue}}
% Define BibTeX command
\def\BibTeX{{\rm B\kern-.05em{\sc i\kern-.025em b}\kern-.08em
    T\kern-.1667em\lower.7ex\hbox{E}\kern-.125emX}}

% Don't print section numbers
\setcounter{secnumdepth}{0}

\setlength{\parindent}{0pt}
\setlength{\parskip}{0pt plus 0.5ex}
%% this changes all items (enumerate and itemize)
\setlength{\leftmargini}{0.5cm}
\setlength{\leftmarginii}{0.4cm}
\setlength{\leftmarginiii}{0.5cm}
\setlist[enumerate,1]{leftmargin=2mm,labelindent=1mm,labelsep=1mm}
\setlist[itemize,1]{leftmargin=2mm,labelindent=1mm,labelsep=1mm}
\setlist[itemize,2]{leftmargin=3mm,labelindent=1mm,labelsep=1mm}
\setlist[itemize,3]{leftmargin=3mm,labelindent=1mm,labelsep=1mm}

%My Environments
\newtheorem{example}[section]{Example}
% -----------------------------------------------------------------------

\begin{document}
\raggedright
\footnotesize
\begin{multicols}{4}


% multicol parameters
% These lengths are set only within the two main columns
\setlength{\columnseprule}{0.25pt}
\setlength{\premulticols}{1pt}
\setlength{\postmulticols}{1pt}
\setlength{\multicolsep}{1pt}
\setlength{\columnsep}{2pt}

\begin{center}
    \fbox{%
        \parbox{0.8\linewidth}{\centering \textcolor{black}{
            {\Large\textbf{CS3211}}
            \\ \normalsize{AY22/23 Sem 2}}
            \\ {\footnotesize \textcolor{myblue}{github.com/SeekSaveServe}}
        }%
    }
\end{center}

% LECTURES

\section{Lectures}
\section{Introduction}
\subsection{L0 and L1}
\textbf{Program Parallelization} \\
\textcolor{blue}{\textbf{Decomposition}}: Decompose a sequential algorithm into tasks (programmer)\\
\begin{itemize}
    \item Granularity of tasks are important
    \item Tasks have dependencies (data or control) between each other which defines the execution order
\end{itemize}
\textcolor{red}{\textbf{Scheduling}}: Assign tasks to processes (programmer / compiler)\\
\textcolor{green}{\textbf{Mapping}} - Map processes to cores (OS)\\

\textbf{Von Neumann Computation Model} instruction and data are stored in memory, and processors computes. \\
\textbf{Memory Wall} disparity between memory speed and processor speed ($\le$ 1 ns VS $\ge$ 100 ns)\\   
\textbf{Processing unit} refers to a core that can execute a kernel thread \\
\textbf{Interconnect} busses betwen different components in the machine \\
\textbf{Node} Machine in a distributed system\\

\textbf{Why Parallel} \\
\textbf{Primary Reasons}
\begin{itemize}
    \item [1] OVercome limits of serial computing
    \item [2] Solve larger problems
    \item [3] Save (wall-clock) time
\end{itemize}
\textbf{Other Reasons}
\begin{itemize}
    \item Take advantage of non-local resources 
    \item Cost/energy saving - use multiple cheaper computing resourcees 
    \item Overcome memory constraints
\end{itemize}

\textbf{Computational Model Attributes} \\
\begin{itemize}
    \item \textbf{Operation mechanism} Primitive units of computation or basic actions of the computer on a specific Architecture 
    \item \textbf{Data Mechanism} How we access and store data in address space 
    \item \textbf{Control Mechanism} How primtive units of computation are scheduled
    \item \textbf{Communication Mechanism} Modes and patterns of exchanging information between parallel tasks (e.g message passing, shared memory)
    \item \textbf{Synchronization Mechanism} ensures to ensure needed information arrives at the right time
\end{itemize}

\textbf{Dependencies and Coordination}
\begin{itemize}
    \item Dependencies among tasks impose constraints on scheduling 
    \item Memory organizations: Shared-memory (threads), distributed-memory (processes) 
    \item Coordination (synchronisation) imposes additional overheads
\end{itemize}

\textbf{Two algorithms}
\includegraphics*[width=7cm]{l1_1.png}
\begin{itemize}
    \item Core 0 is active throughout the execution
    \item Some cores are idle
    \item This is a lot better than having all cores idle while the master core is executing
\end{itemize}


\textbf{Parallel Performance} 
\begin{itemize}
    \item Execution time Vs Throughput
    \item Parallel execution time = computation time + parallelization overheads 
    \item Overheads: Distribution of work(tasks) to porocesses, information exchange, synchronisation, idle time, etc
\end{itemize}

\section{Background on Parallelism}
\subsection*{L2: Processes and Threads}
\textbf{Process}
\begin{itemize}
    \item Identified by PID 
    \item Program counter, global data (open files, network connections), stack or heap, current values of the registers (GPRs and Special)
    \item These information are abstracted in the PCB, and each proecss can be viewed as having exclusive access to tis address space 
    \item Explicit communication is needed
    \item \textbf{Disadvantage}
    \begin{enumerate}
        \item High overhead of system calls
        \item Potential re-allocation of data-structures
        \item Communication goes through OS (system calls) and context switch is costly
    \end{enumerate}
\end{itemize}

\includegraphics*[width=7cm]{memory_space.png}

\textbf{Multi tasking}
\begin{itemize}
    \item Overhead: Context switching (PCB change) is needed and states of suspended process must be saved 
    \item Time slicing: Pseudo-parallelism
    \item Child processes can use parent's data
\end{itemize}

\textbf{Inter-process communication (IPC)}
\begin{itemize}
    \item Shared memory: need to protect access with locks 
    \item Message passing: Blocking, unblocking, Synchronous, unsynchronous
\end{itemize}

\includegraphics*[width=7cm, height=3.8cm]{except_interrupt}

\textbf{Threads}
\begin{itemize}
    \item A process may have multiple indepedent control flows called threads 
    \item Each thread has its own stack and registers (PC, SP, registers), but share the same address space 
    \item Shared memory model and Shared memory architecture
\end{itemize}



\section*{Architecture}
\subsection*{L3: Processor and memory organization}


\subsection*{L7: Cache coherence and memory consistency}


\subsection*{L11: Interconnection networks}


\section*{Parallel Computation Models}
\subsection*{L4: Shared-memory programming models}


\subsection*{L6: Data parallel models (GPGPU)}


\subsection*{L9,10: Distributed-programming models}


\section*{Performance and Scalability of Parallel Programs}
\subsection*{L5: Performance of parallel systems}

\subsection*{L8: performance instrumentation}


\section*{New Trends}
\subsection*{L12: Energy efficient computing}


% Tutorials


% Misc

\end{multicols}
\end{document}