\documentclass[10pt, landscape]{article}
\usepackage[scaled=0.92]{helvet}
\usepackage{calc}
\usepackage{graphicx}
\usepackage{multicol}
\usepackage{ifthen}
\usepackage[a4paper,margin=3mm,landscape]{geometry}
\usepackage{amsmath,amsthm,amsfonts,amssymb}
\usepackage{color,graphicx,overpic}
\usepackage{hyperref}
\usepackage{newtxtext} 
\usepackage{enumitem}
\usepackage{graphicx}
\usepackage[table]{xcolor}
\usepackage{mathtools}
\usepackage[document]{ragged2e}
\usepackage{listings}
\setlist{nosep}
\usepackage{subfig}


% for including images
\graphicspath{ {./resources/} }


\pdfinfo{
  /Title (CS2109S.pdf)
  /Creator (TeX)
  /Producer (pdfTeX 1.40.0)
  /Author (Pei Cheng Yi)
  /Subject (CS3211)
  /Keywords (CS3211, nus,cheatsheet,pdf)}

% Turn off header and footer
\pagestyle{empty}

\newenvironment{tightcenter}{%
  \setlength\topsep{0pt}
  \setlength\parskip{0pt}
  \begin{center}
}{%
  \end{center}
}

% redefine section commands to use less space
\makeatletter
\renewcommand{\section}{\@startsection{section}{1}{0mm}%
                                {-1ex plus -.5ex minus -.2ex}%
                                {0.5ex plus .2ex}%x
                                {\normalfont\large\bfseries}}
\renewcommand{\subsection}{\@startsection{subsection}{2}{0mm}%
                                {-1explus -.5ex minus -.2ex}%
                                {0.5ex plus .2ex}%
                                {\normalfont\normalsize\bfseries}}
\renewcommand{\subsubsection}{\@startsection{subsubsection}{3}{0mm}%
                                {-1ex plus -.5ex minus -.2ex}%
                                {1ex plus .2ex}%
                                {\normalfont\small\bfseries}}%
\renewcommand{\familydefault}{\sfdefault}
\renewcommand\rmdefault{\sfdefault}
%  makes nested numbering (e.g. 1.1.1, 1.1.2, etc)
\renewcommand{\labelenumii}{\theenumii}
\renewcommand{\theenumii}{\theenumi.\arabic{enumii}.}
\renewcommand\labelitemii{•}
\renewcommand\labelitemiii{•}
%  convenient absolute value symbol
\newcommand{\abs}[1]{\vert #1 \vert}
%  convenient floor and ceiling
\newcommand{\floor}[1]{\lfloor #1 \rfloor}
\newcommand{\ceil}[1]{\lceil #1 \rceil}
%  convenient modulo
\newcommand{\Mod}[1]{\ \mathrm{mod}\ #1}
%  for logical not operator, iff symbol, convenient "if/then"
\renewcommand{\lnot}{\mathord{\sim}}
\let\then\Rightarrow
\let\Then\Rightarrow
%  vectors
\newcommand{\vv}[1]{\boldsymbol{#1}}
\newcommand{\VV}[1]{\overrightarrow{#1}}
%  column vector
\newcommand{\cvv}[1]{\left(\begin{smallmatrix}#1\end{smallmatrix}\right)}
\newcommand{\code}[1]{\textcolor{myblue}{\texttt{#1}}}
\newcommand\bggreen{\cellcolor{green!10}}

\makeatother
\definecolor{myblue}{cmyk}{1,.72,0,.38}
\everymath\expandafter{\the\everymath \color{myblue}}
% Define BibTeX command
\def\BibTeX{{\rm B\kern-.05em{\sc i\kern-.025em b}\kern-.08em
    T\kern-.1667em\lower.7ex\hbox{E}\kern-.125emX}}

% Don't print section numbers
\setcounter{secnumdepth}{0}

\setlength{\parindent}{0pt}
\setlength{\parskip}{0pt plus 0.5ex}
%% this changes all items (enumerate and itemize)
\setlength{\leftmargini}{0.5cm}
\setlength{\leftmarginii}{0.4cm}
\setlength{\leftmarginiii}{0.5cm}
\setlist[enumerate,1]{leftmargin=2mm,labelindent=1mm,labelsep=1mm}
\setlist[itemize,1]{leftmargin=2mm,labelindent=1mm,labelsep=1mm}
\setlist[itemize,2]{leftmargin=3mm,labelindent=1mm,labelsep=1mm}
\setlist[itemize,3]{leftmargin=3mm,labelindent=1mm,labelsep=1mm}

%My Environments
\newtheorem{example}[section]{Example}
% -----------------------------------------------------------------------

\begin{document}
\raggedright
\footnotesize
\begin{multicols}{4}


% multicol parameters
% These lengths are set only within the two main columns
\setlength{\columnseprule}{0.25pt}
\setlength{\premulticols}{1pt}
\setlength{\postmulticols}{1pt}
\setlength{\multicolsep}{1pt}
\setlength{\columnsep}{2pt}

\begin{center}
    \fbox{%
        \parbox{0.8\linewidth}{\centering \textcolor{black}{
            {\Large\textbf{CS3211}}
            \\ \normalsize{AY22/23 Sem 2}}
            \\ {\footnotesize \textcolor{myblue}{github.com/SeekSaveServe}}
        }%
    }
\end{center}

% LECTURES

\section{Lectures}
\subsection{L1: Introduction to Concurrency}
\textbf{Concurrency} \\ 
\begin{itemize}
    \item Concurency is pervasive when modern computers have several cores and types of memory
    \item $\ge$ 2 activities making progress at the same time (overlaping time periods)
    \item Involves interleaving of instructions from different activities
\end{itemize}

\textbf{Parallelism} \\
\begin{itemize}
    \item $\ge$ 2 processes executing and making progress \textit{simultaneously} 
    \item Hardware dependent: requires hyperthreading (SMT), or multi-core and hardware threads
\end{itemize}

\textbf{Processes Vs threads} \\
\begin{itemize}
    \item Independent Vs Shared memory (address) space 
    \item Both use independet stack 
    \item Expensive Vs Cheap context switch 
    \item OS facilitated vs Non OS facilitated inter-process/thread Communication
    \item Expensive (copy on write mediates this somewhat) Vs Cheap creation
\end{itemize}

\textbf{Interrupts} \\
\begin{itemize}
    \item Asynchronous (independent to program execution) 
    \item Used by OS to interact with the programme
    \item Triggered by external events (e.g. I/O, timer, hardware failure)
\end{itemize}

\textbf{Exceptions} \\ 
\begin{itemize}
    \item Synchronous (dependent on program execution)
    \item Used by process to interact with the OS
    \item Triggered by process error (e.g. underflow, overflow)
\end{itemize}

\textbf{User thread} \\ 
\begin{itemize}
    \item Library created, linked to one kernel thread
\end{itemize}

\textbf{Race condition} \\
\begin{itemize}
    \item Outcome depends on reltive ordering of operations on $ge$ 2 Threads
    \item a flaw that occurs when the timing or ordering of events affects a program's correctness
\end{itemize}

\textbf{Data Race} \\ 
\begin{enumerate}
    \item $\ge$ 2 concurrent threads concurrently access a shared resource without Synchronisation / fixed ordering
    \item At least one modifies shared resource
    \item Causes undefined behaviour
\end{enumerate}

\textbf{Mutex} \\ 
\begin{itemize}
    \item Creates critical section can be treated as a large atomic blocks
    \item Only one thread at a a time
    \item Supported by a hardware instruction (CAS, test and set etc)
    \item \textbf{Properties:} Mutex, progres, bounded wait, performance 
    \item Provides \textbf{serialisation} (less concurrency)
\end{itemize}

\textbf{Critical section} \\ 
\begin{itemize}
    \item Safety: nothing bad happens
    \item Liveness: Something good (progress) happens 
    \item Performance: depends on aggregate performance of all threads
\end{itemize}

\textbf{Locks} \\ 
\begin{itemize}
    \item Primitive that is provided b the hardware, minimal semantic 
    \item E.g. Test and set 
\end{itemize}

\textbf{Deadlock iff} \\ 
\begin{enumerate}
    \item Mutex: One resource held in a non-shareable state
    \item Hold and wait: One proces holding one resource and waiting for another resource 
    \item No-preemption: Resource and critical section cannot be aborted externally 
    \item Circular wait
    \item Note: Lock free can deadlock
\end{enumerate}

\textbf{Dealing with deadlock} \\
\begin{itemize}
    \item Prevention: Eliminate one of the above conditions (E.g. hold all locks at the start)
    \item Detection and recovery: Look for cycles in dependencies (E.g. wait for graph)
    \item Avoidance: Control allocation of resources
\end{itemize}

\textbf{Starvation} \\ 
\begin{itemize}
    \item One process cannot progress becausee another process is holding on a resource it needs
    \item Side effect of scheduling algorithm 
    \item Wait-die and wound-wait are possible solutions, if priority of processes is preserved
\end{itemize}

\textbf{Advantages of concurrency} \\ 
\begin{itemize}
    \item Performance
    \item Separation of concerns
\end{itemize}

\textbf{Disadvanageous of concurrency} \\ 
\begin{itemize}
    \item Maintenance and debugging
\end{itemize}

\textbf{Task parallelism} \\ 
\begin{enumerate}
    \item Do the \textbf{same type of} work faster
    \item Task dependency graph can be parallel
    \item Make tasks specialists: Same type of tasks are assigned to the same thread
    \item Divide a sequence of tasks among threads to solve complexed task 
    \item \textbf{Pipeline:} 1 type of thread for one phase of execution
\end{enumerate}

\textbf{Data parallelism} \\ 
\begin{enumerate}
    \item Do \textbf{more work} in the same amount of time
    \item Divide data to chunks and execute by different threads
    \item Embarasingly parallel tasks
\end{enumerate}

\textbf{Challenges of concurrency} \\
\begin{enumerate}
    \item Finding enough parallelism: Amadahl's law
    \item Granularity of tasks
    \item Locality
    \item Coordination and Synchronisation
    \item debugging
    \item Performance and monitroring
\end{enumerate}

\subsection*{L2: Tasks, threads, synchronisation in modern C++}
\textbf{History of CPP} \\
\begin{itemize}
    \item 1998: No support for multithreading
    \begin{enumerate}
        \item Effects of language model are assumed to be sequential and there are no established memory model
        \item Different libraries used different memory models
        \item Execution threads were not acknowledged
    \end{enumerate}
    \item 2011: C++11
    \begin{enumerate}
        \item Standard threads are implemented
        \item Thread aware memory model.Do not rely on platform specific extensions to guarantee behaviour
        \item Atomic operations library, class to manage threads, protected shared data etc. 
    \end{enumerate}
\end{itemize}

\textbf{Four ways to manage threads} \\ 
\begin{enumerate}
    \item Declare a function that returns a thread

    \begin{lstlisting}[language=c++,breaklines=true, breakatwhitespace=true]
void hello() {
    std::cout << "Hello Concurrent World\n";
}
int main() {
    std::thread t(hello);
    t.join();// existing thread waits for t to finish
}
    \end{lstlisting}

    \item Thread with a function object 
    \begin{lstlisting}[language=c++, breaklines=true, breakatwhitespace=true]
class background_task {
public: 
    void operator()() const {
        do_something();
        do_something_else();
    }
};
/* Calleable object */
background_task f;
std::thread my_thread(f);
    \end{lstlisting}
    \begin{itemize}
        \item  $std:: my\_thread(background\_task())$ declares a function that takes a single parameter (type *f() $\rightarrow$ object)
        \item This is not the same as using a function object!
    \end{itemize}
\item Threads with a lambda expression (local fn instead of a calleable object) 
\begin{lstlisting}[language=c++, breaklines=true, breakatwhitespace=true]
std::thread my_thread([]{
    do_something();
    do_something_else();
});
\end{lstlisting}
\end{enumerate}

\textbf{Wait} \\ 
\begin{itemize}
    \item Uses join() on the thread instance exactly once 
    \item Use joinable to check
    \item Local variables do not go out of scope 
    \item Blocking
\end{itemize}

\textbf{Detach()} \\ 
\begin{itemize}
    \item Local variable passed might go out of scope and 'disappear' during runtime, causing invalid access for the detached thread 
    \item Example 
    \begin{lstlisting}[language=c++, breaklines=true, breakatwhitespace=true]
void oops(){
    int local_state = 0;
    /* Reference passed might become invalid */
    func my_func(local_state);
    std::thread my_thread(my_func);
    my_thread.detach();
} /*oops ends here and local_state will be destroyed */
    \end{lstlisting}
    \item Not blocking 
\end{itemize}


\textbf{Passing arguments} \\
\begin{enumerate}
    \item by value $std::thread t(f, 3, "hello")$
    \item by reference $std::thread t(f, 3, buffer)$
    \begin{itemize}
        \item Buffer is a charbuffer that only gets converted to str when we call f
        \item Hence it is possible for buffer to go out of scope 
        \item Fix: Use explicit cast $std::thread(f, 3, std::string(buffer))$ 
        \item \textbf{Major issue} with passing by reference is that threads outside of the scope can use it in \textbf{unsafe} ways. E.g. Not using mutex on shared data, deletion etc 
    \end{itemize}

    \item by copy
    \begin{lstlisting}[language=c++, breaklines=true, breakatwhitespace=true]
void update_data_for_widget(widget_id w, widget_data& data);
void oops_again(widget_id w) {
    widget_data data;
    /* a copy of data is passed */
    std::thread t(update_data_for_widget, w, data);
    display_status();
    t.join();
    /* changes made to the copy is not reflected to other threads */
    process_widget_data(data);
}
    \end{lstlisting}
    \begin{itemize}
        \item Fix: use reference $std::thread t(update_data_for_widget, w, std::ref(data))$
    \end{itemize} 
\end{enumerate}



\textbf{Ownership in C++} \\ 
\begin{itemize}
    \item Owner is an object containing a pointer to an object allocated by $new$ for which the owner is responsible for deleting
    \item Every object on free store (heap, dynamic store) must have \textbf{exactly one} owner
\end{itemize}

\textbf{C++ Resource Management} \\ 
\begin{itemize}
    \item For scoped objects, desctructor is implicit at scope exit 
    \item Free store objects (created using $new$) requires explicit delete
\end{itemize}


\textbf{RAII} \\ 
\begin{itemize}
    \item Binds the lifetime of a resource that must be acquired before use to the lifetime of an object
\begin{lstlisting}[language=C++, breaklines=true, breakatwhitespace=true]
/* Handle interrupts using RAII */
void enqueue(Job job) {
    std::unique_lock lock{mut}; // constructor locks mutex
    jobs.push(job);// destructor unlocks mutex
}
\end{lstlisting}

\textbf{Lifetime} \\
\begin{itemize}
    \item Lifetime begins when storage is obtained and its initialization is complete (except std::allocator::allocate)   
    \item Lifetime ends when :
    \begin{itemize}
        \item Non-class type (int): destroyed
        \item Class type: When destructor is called 
        \item Reference: begins with initialisation and ends when destroyed. A dangling reference is possible.
    \end{itemize}
\end{itemize}

\textbf{Ownership of thread} \\ 
\begin{itemize}
    \item Moveable but not copyable
    \begin{lstlisting}[language=C++, breaklines=true, breakatwhitespace=true]
void some_function();
void some_other_function();
std::thread t1(some_function);
/* t1 no longer references the thread */
std::thread t2 = std::move(t1); 
/* t1 now owns a new thread */
t1 = std::thread(some_other_function);
std::thread t3;
/* t3 owns the thread running some function */
t3 = std::move(t2);
/* t1 already owns a thread, this will trigger a runtime error */
t1 = std::move(t3); 
    \end{lstlisting}
    \begin{itemize}
        \item C++ compiler cannot catch this
    \end{itemize}
    \item Ownerhship can be moved out of a function and moved into another function
    \begin{lstlisting}[language=C++, breaklines=true, breakatwhitespace=true]

/* Transferrring out of a function */
std::thread g() {
    void some_function();
    std::thread t(some_function);
    return t; // ownership transferred out of g()
}

/* Transferring into a function */
void f(std::thread t);
void g() {
    void some_other_function();
    std::thread t(some_other_function);
    f(std::move(t)); // ownership transferred into f()
}
    \end{lstlisting}
\end{itemize}


\textbf{Mutex in C++} \\
\begin{itemize}
    \item $std::lock_guard$ locks the mutex upon initialisation, unlocks upon destruction
    \item $std::lock_guard<std::mutex>{some_mutex};$ 
    \item Group mutex and protected data together in a class rather than use global variables
    \item Never pass data or pointers (via returns, sotring in externally visible memory, as input to functions etc) when their usage is not guaranteed to be safe
\end{itemize}


\textbf{Types of lock guards}
\begin{itemize}
    \item \textbf{lock guard} no manual lock, can lock many or one mutex at once without deadlock
    \item \textbf{Scoped lock} accepts and locks a list of mutexes. Can be unintentionally initialized without a mutex
    \item \textbf{unique lock} manual unlock, defers locking using $std::defer_lock$, only single mutex. 
\end{itemize}

\textbf{Condition Variable} \\
\begin{itemize}
    \item Use condition variables to wait for a an event to be triggered by another thread
    \item Avoids busy waiting 
    \begin{lstlisting}[language=C++, breaklines=true, breakatwhitespace=true]
std::condition_variable.wait(lock, []{return predicate});
    \end{lstlisting}
    \item if condition is satisfied, returns
    \item unlocks the mutex and places the thread in block state if condition is not satisfied
    \item $std::condition_variable.nofiy_one();$ to notify one thread waiting on the cond
\end{itemize}


\textbf{Spurious Wake} \\ 
\begin{itemize}
    \item Thread wakes up from waiting, but is blocked again as the resource required is not available
    \item Leads to unnecessary context switching
    \item Use conditionals to prevent spurious wake
\end{itemize}

\end{itemize}

\textbf{Shared DS - Invariants} \\ 
\begin{itemize}
    \item Often broken during an update 
    \item \textbf{Case study: Doubly LL}
\includegraphics*[width=7cm, width=7cm]{dll.png}
    \item  invariant is temporarily broken during an update, and we need to prevent objects from accessing the DS during this time
\end{itemize}

\subsection{L3: Atomics and Memory Model in C++} 

\textbf{Reordering of operations} \\ 
\begin{itemize}
    \item Compiler may reorder (potentially conflicting) actions for performance 
    \item Not visible to programmers
\end{itemize}

\textbf{As-if rule} \\
\begin{itemize}
    \item Reordering is allowed as long as:
    \begin{enumerate}
        \item At termination, data written to files is exactly as if the program was executed as written (same final state)
        \item Prompting text that is sent to interactive devices will be shown before the program waits for input
        \item Programs with undefined behaviour is exempted from these rules.
    \end{enumerate}
\end{itemize}

\textbf{Multi-threading aware memory model} \\ 
\begin{itemize}
    \item Using synchronisation constructs (mutexes, barriers etc) should preclude the need for a memory model since they serialise threads 
    \item Memory model gives us more flexibility and speed by getting us closer to the machine
\end{itemize}

\textbf{Stucture of memory model} \\
\begin{itemize}
    \item Every object has a memory location, some occupy exactly one, some occupy many 
    \item The changes in memory location / what is stored there affects other threads
\end{itemize}

\textbf{Modification Order} \\ 
\begin{itemize}
    \item Compose of all writes to an object from all threads in the program
    \item MO varies between runs, each object has their own MO
    \item The programmer is responsible that threads agree on the MO (if not, race condition happens)  
\end{itemize}

\textbf{MO - Requirements} \\ 
\begin{itemize}
    \item The MO of each object is monotonic within a thread 
    \item But the relative ordering of MO of different objects is not guaranteed
\end{itemize}

\textbf{MO - Building Blocks} \\
\begin{itemize}
    \item \textbf{Sequenced-before (SB)}
    \begin{itemize}
        \item Each lines of code in a thread is sequenced before the next line
        \item There is \textbf{NO} sequenced before in a statement with many function calls 
    \end{itemize}
    \item \textbf{Synchronises-with (SW)}
    \begin{itemize}
        \item Established by a $load$ from $T_i$ reading $T_j$'s $store$ 
        \item Both $T_i$ and $T_j$ are synced with respect to the common value in the MO
    \end{itemize}
    \item \textbf{Happens-before (HB)} When an operation happens before another operation due to SW or SB
    \item \textbf{Interthread Happens Before (IHB)}
    \begin{itemize}
        \item When a $store$ in $T_i$ established a sequenced before a $load$ in $T_j$, $sotre_i$ happens before $load_j$ 
        \item IHB $\subseteq$ HB
    \end{itemize}
    \item \textbf{Visible Side effects} 
    \begin{itemize}
        \item Side effect of write A on O is visible to a read B on O if:
        \begin{enumerate}
            \item A HB B
            \item There is no other side effects to O that happens between A and B
        \end{enumerate}
        \item If the side effect of A is visible to B then the longest contiguous subset of the side-effects to O (that B does not HB) is known as the visible sequence of side effects 
    \end{itemize}
    \item \textbf{Modification Order}
\end{itemize}

\textbf{MO - Seq Const} \\
\begin{itemize}
    \item The default
    \item All threads must see the same ordering of operation
    \item Synchronises with a sequentially consistent load of the same variable that reads the value loaded 
    \item Does not apply to atomic operations with relaxed ordering
    \item Performance penalty when working with weakly ordered machine instructions (common) 
    \item Essentially a serailised monoversion - global total order enforced    
    \item Only guaranteed for data-race free programs (which is difficult since C++ is not as safe as Rust)
\end{itemize}

\textbf{MO - Relaxed} \\ 
\begin{itemize}
    \item Atomic operations don't conform with SW relationships
    \item Happens before still applies within the thread $\rightarrow$ monotonicity and SB within the thread is preserved
    \item No HB between load and store, different store operations from T1 can be viewed out of order by reads in T2
    \item T1: $x=1, y=0$. T2 can see y=0 without seeing x=1 since there is no SW between the two threads even though x=1 HB y=1 in T1.  
\end{itemize}


\textbf{MO - Acquire Release} \\
\begin{itemize}
    \item No total modification order, but there is a partial order
    \item Read - acquire updates about the memory order, load - release updates about the memory order
    \item A link between acquire and release acts like a barrier 
\end{itemize}


\textbf{MO - Mixing Models} \\
\begin{itemize}
    \item Seq const and Release Acquire: load and store of seq const behaves similar to release acquite 
    \item any MO and relaxed: Relaxed behaves like relaxed but is bounded by the other more limiting MO 
    \begin{lstlisting}[language=C++]
// T1
x.store(true, std::memory_order_relaxed);
y.store(true, std::memory_order_release);
// T2
while (!y.load(std::memory_order_acquire) );
/* Never fires because acquire and release */
/* x.store HB y.store  & y.store SW y.load */
assert(x.load(std::memory_order_relaxed)); 
    \end{lstlisting}
\end{itemize}

\textbf{Atomic Operations} \\ 
\begin{itemize}
    \item Compiler ensures necesary synchronisation is in place and enforces MO 
    \item Atomic ops are indivisible
    \item Atomic load loads either the initial value or the value stored by one of the modifications (cannot be half-done)
    \item Can be lock free or be implemented using mutex (which wipes off performance gains)
    \item Not necessarily race free
\end{itemize}

\textbf{Fences} \\ 
\begin{itemize}
    \item Enforce MO constraints without modifying data, typically combined with atomic operations that uses relaxed MO
    \item Memory barriers: places a line in code that cannot be crossed 
    \item E.g. atomic thread fence with memory order release prevents preceeding reads and writes from moving past subsequent stores
\end{itemize}

\pagebreak
% TUTORIALS
\section{Tutorials}
\subsection*{T1: Threads and Synchronisation}
\textbf{Why mutexes work - standard argument}\\ 
\begin{itemize}
    \item Define a critical section that contains all acesses to the shared resource 
    \item Argue that mutex guarantees mutual exclusivity of threads - removes interleaving, data race precluded
\end{itemize}

\textbf{Why mutexes work - theoretical argument}\\
\begin{itemize}
    \item Lock and unlock appears in a single total order
    \item Only one thread owns the lock at any pointer
    \item Unlock happens after lock, creating a synchronises with relationship between processes and \textbf{serialises the interleaving} - no concurrent access
\end{itemize}

\textbf{Monitor} \\
\begin{itemize}
    \item Allows us to block until a condition becomes true
    \item The monitor has: 
    \begin{enumerate}
        \item A mutex on the critical section
        \item A condition variable
        \item A condition to wait for 
    \end{enumerate}
    \begin{lstlisting}[language=C++, breaklines=true, breakatwhitespace=true]
std::condition_variable cond;

Job dequeue() {
    std::unique_lock lock{mut};
    /* wait until there is a job */
    cond.wait(lock, [this]() { return !jobs.empty(); });
    Job job = jobs.front();
    jobs.pop();
    return job;
    }

void enqueue(Job job) {
    {
        std::unique_lock lock{mut};
        jobs.push(job);
    }
    /* notify one thread waiting on the condition variable */
    cond.notify_one();
    }
    \end{lstlisting}
\end{itemize}

\pagebreak
% USEFULE FACTS
\subsection*{Classical Synchronisation Problems}

\subsection*{Comparison between Rust, Go, C++}
\textbf{Ownership} \\ 
\begin{itemize}
    \item C++ has RAII to manage resources, moveable but not copyable reference
\end{itemize}

\end{multicols}
\end{document}